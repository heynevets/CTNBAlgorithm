\section{Theoretical Bases and Literature Review}


%\subsection{definition of the problem}
The problem we are trying to tackle is the identification of topics that have limited support on Stack Overflow.

\subsection{Theoretical background of the problem}
Topic identification in natural language processing is commonly done with Latent Dirichlet Allocation (LDA). LDA is an unsupervised clustering algorithm for natural language processing. It identifies primary topics based on the assumption that each document is composed of a mixture of topics that generate keywords. LDA matches frequently co-occurring keywords by iteratively improving its guesses. Once the solution converges, humans can manually label each topic based on its keywords. 
\subsection{Related research to solve the problem}
A large body of researchers have investigated the nature of Stack Overflow. For example, Cristoffer Rosen and Emad Shihab have researched what mobile developers were asking on Stack Overflow \cite{N1_Paper}. Rosen and Shihab used Latent Dirichlet allocation (LDA) to cluster questions and then examined various statistics related to those questions such as response time, average number of answers, or the ratio between question views and the percentage of questions with accepted answers. Other papers have also implemented LDA, such as Manipal University’s examination of question and answer quality \cite{N2_Paper}. Other work exploring the non-functional requirements for developers on Stack Overflow have also used LDA to cluster questions and then applied their analysis on each topic. As we can see, there is a large body of research using LDA to study Stack Overflow questions.
Research is done not only on questions on Stack Overflow, but also the answers. Calefato et al. have done extensive research on understanding what makes a good answer \cite{N3_Paper}. They evaluated a good answer as the answer accepted by the asker as acceptable. They used Alternating Decision Trees (ADT), a binary classifier, to classify responses as either accepted or not. Using various features of the data, they were able to achieve 90\% accuracy in classifying answers. They also found that counting number of votes on the answers alone is a better (with 93\% accuracy) yet simpler metric to predict the accepted answer.
\subsection{Advantage/disadvantage of those research}
	One advantage of the research using LDA is that it naturally finds topics that people are asking about on Stack Overflow. This reduces human bias in clustering questions into topics. LDA is also generally better than TF-IDF in terms of parsing the data. One advantage of using just the question titles is that it reduces the noise in each topic. One disadvantage though is that some questions can be misclassified since it doesn’t include information contained in the body of the question. It was also helpful that the authors examined various features of the data to compare topics, such as the ratio of views to the percentage of accepted responses. These statistics allow us to see trends and draw insights about each different topic. However, the statistics analyzed only go so far, and there is information to be mined that hasn’t been explored yet. An advantage of the ADT that they used to analyze the responses is that they were able to draw insights from the answers alone. However,  the disadvantage is that for Stack Exchange sites, there is no use of such method, user votes for answers are proved a more efficient and more accurate way to evaluate the quality of answers.
	
\subsection{Solution to solve this problem}
	Our solution to identifying topics that have limited support on Stack Overflow is a multi-step process. First, we will identify topics using LDA. Then, we will examine statistics related to each topic to try to identify which topics have the least support. We will identify such topics by evaluating “supply” vs. “demand”. We wish to examine various metrics to analyze the “supply” and “demand” from the questions. We will see if characterizing demand as the number of question  views, the number of question upvotes, or the number of questions allows us to draw insights not seen previously. Similarly, we will examine characterizing supply as the number of answers, the number of answer upvotes, or the number of unique responders allows us to see insights in the data.\\
%\subsection{where your solution different from others}
	Our solution differs from the papers examined in the ways that we evaluate whether topics have limited support. While some of the papers did broad analysis on a wide range of topics, or have done detailed analysis on various other features of the data, only we will provide such a detailed examination of the supply and demand in the questions and answers. An especially unique feature we want to examine is the number of unique responders, since few responders indicate that there are only a few experts in the field.\\
%\subsection{why your solution is better}
	Our solution is better because we examine statistics not previously analyzed, statistics that are more relevant in determining which topics have limited support on Stack Overflow. Our utilization of data previously unexplored will allow us to draw better insights into Stack Overflow and the developer community at large.


